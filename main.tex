\documentclass{article}
\usepackage{textcomp}
\newcommand{\perthousand}{\textperthousand}
\usepackage{graphicx}
\usepackage{amsmath}
\usepackage{amssymb}
\usepackage{braket}
\usepackage[italicdiff]{physics}
\usepackage{gensymb}
\usepackage{hyperref}
\usepackage{enumerate}
\newcommand{\m}[1]{_{\mathrm{#1}}}
\usepackage[utf8]{inputenc}
\usepackage[T1]{fontenc}
\usepackage{mathtools}
\usepackage[thinc]{esdiff}
\usepackage{float}


\title{Week 1 OAF Quant}
\author{Harik Sodhi}
\date{October 26 2025}

\begin{document}
\maketitle
\section{Value}
Why might something have value?
- Supply and Demand
- Necessity
- Ability to sell in the future
- Investment possibility
\section{Time Value of Money - Present Value}
$PV=\frac{FV}{(1+r)^T}$ where FV is future value, PV is current value, $r$ is interest rate.
\newline
Can increase $r$ in high risk cases to reduce PV
\section{Liquidity}
You'll get a worse price with large orders otherwise you go too deep into the order book, which makes value worse. This requires splitting into smaller orders or do it Over the Counter (OTC).
\section{Efficient Market Hypothesis}
Markets are efficient - participants push price towards a fundamentally correct level given all publically available information. Market movements are only due to new information.
\section{Market Makers}
- Provide Liquidity
- Profit from the bid-ask spread
\section{Basic Models}
A model is just some kind of mathematical framework you use to represent or simplify something.
Suppose apple prices are going to 1.20 or 0.80 and 50$\%$ of people know all the information. The others have no information.
\section{Volatility}
Some products' prices tend to move aroudn much more than others.
This is defined as the standard deviation of returns.
\section{Sharpe Ratio}
$Sharpe Ratio=\frac{R_p - R_f}{\sigma_p}$ where $R_p$ is portfolio return, $R_f$ is risk free return, $\sigma_p$ is standard deviation of portfolio.
\section{Correlation}
Putting all your money in correlated stuff is very bad.
\section{Correlation}
$$Corr(X,Y)=\frac{E(XY)-E(X)-E(Y)}{\sqrt{Var(X)Var(Y)}}$$
\section{Hedging}
Hedge against the market. In general, you should aim for your return to be independent of the market. You can short multiple stocks that are close to the rest of the market.
\section{Some more models}

\end{document}